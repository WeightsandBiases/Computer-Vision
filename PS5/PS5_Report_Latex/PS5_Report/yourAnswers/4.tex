Tracking through occlusions with image resizing was particularly difficult.\\
I broke down the problem into two sets of challenges and solved them separately\\
1) The first challenge was that of resizing the template.\\
As the lady moves away from the camera, the image of her gets smaller.\\
To solve this challenge, we add uniform randomness to the size of\\
our tracking template to the image as a third dimension along with the\\
x and y positions of her in the image.  As the particles track and are\\
resampled, the particles with the best template size will get resampled\\
more often.\\
\\
2) The second challenge was that of occlusion.  Two individuals move into\\
the camera and occlude our target from being tracked. To solve the occlusion\\
challenge the algorithm must first detect occlusion, this is done using\\
a simple threshold of the mean squared error.  If the mse of the  template\\
and the particles are too great, occlusion is detected. When occlusion \\
is detected, the algorithm will have lower dynamics (where particles will fan\\
out and search for the target) and resampling and calculation of weights are\\
avoided completely\\
\\
By combining these two methods, the algorithm was able to track the target\\
for part 4\\

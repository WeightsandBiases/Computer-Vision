Part 5 was solved using a modification of the MD particle filter from part 4\\
As the algorithm can detect and respond to occlusions of the target.\\
Particle filters were selected in part 5 because the templates of the targets\\
are changing as they are moving and require multiple samples (particles) to\\
track. (For example the third target has a white shirt that is later occluded\\
in the scene as he rotates his body). In order to track our three targets\\
three particle filters were used.  A template was taken for each target\\
when they entered the scene, and the particle filter would be stopped at \\
the frame the target exits the scene.  This is somewhat of a knowledge based\\
approach as it requires knowlege of when the target will enter a scene. A \\
more general solution would be to use mse thresholds, as was done in part 4\\
to detect entry and exits of targets in the scene.  Overall, this assignement\\
taught me the challenges and intricacies of computer vision and I have gained\\
I can only imagine trying to do this to create snapchat filters or animojis\\
I have gained greater appreciation for this ever advancing field.\\
